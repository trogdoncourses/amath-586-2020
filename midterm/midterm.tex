\documentclass[10pt]{amsart}
\usepackage[margin=1.5in]{geometry}
\usepackage{amssymb,amsmath,enumitem,url}
\usepackage{listings}
\lstset{basicstyle=\ttfamily}


\DeclareMathOperator{\D}{d}
\DeclareMathOperator{\E}{e}
\newcommand{\half}{\frac{1}{2}}
\newcommand\unp{U^{n+1}}
\newcommand\unm{U^{n-1}}
\newcommand{\bigo}{{\mathrm O}}
\newcommand{\reals}{\mathbb R}



\begin{document}

%\topmargin -1.0in
%\textheight 10.5in
\pagestyle{empty}

\newcommand{\mline}{\vspace{.2in}\hrule\vspace{.2in}}


\title{\bf { AMATH 586 Spring 2020 \\ Midterm Exam ---
Due May 15 on GitHub by 11pm} }
\maketitle
\centerline{Be sure to do a {\tt git pull} to update your local version of the {\tt amath-586-2020} repository.}

This entire exam concerns system of ODEs called the \emph{Toda lattice}.  The system is defined using positions $p_j$, $j = 0, \pm 1, \pm 2,\ldots$ and momenta $q_j$, $j = 0, \pm 1, \pm 2, \ldots$:
\begin{align}\label{eq:toda}
  \begin{split}
  q_j'(t) &= p_j(t),\\
  p_j'(t) &= e^{-(q_{j+1}(t) - q_j(t))} - e^{-(q_{j}(t) - q_{j-1}(t))}, \quad j = 0, \pm 1, \pm 2,\ldots.
  \end{split}
\end{align}
This is, as defined, an infinite-dimensional ODE system.  We will consider finite-dimensional approximations. 


\mline


\begin{enumerate}[label={\bf Problem~{\arabic*}:}]
  \item   Define new variables
  \begin{align}\label{eq:toda_cov}
    \begin{split}
    a_j(t) &= \frac 1 2 e^{-(q_{j+1}(t) - q_j(t))/2},\\
    b_j(t) &= - \frac 1 2 p_j(t), \quad j = 0, \pm 1, \pm 2,\ldots.
    \end{split}
  \end{align}
  Using \eqref{eq:toda_cov}, first show that the Toda lattice \eqref{eq:toda} can be written as
  \begin{align} \label{eq:toda_ab}
    \begin{split}
    a_j'(t) &= a_j(t) (b_{j+1}(t) - b_j(t)), \\
    b_j'(t) &= 2 (a_{j-1}^2(t) - a_{j}^2(t)),  \quad j = 0, \pm 1, \pm 2,\ldots.
    \end{split}
  \end{align}
Now consider the time-dependent tridiagonal matrices:
  \begin{align*}
    T(t) = \begin{bmatrix}
      \ddots & \ddots\\
      \ddots & b_{j-1}(t) & a_{j-1}(t)\\
      & a_{j-1}(t) & b_{j}(t) & a_{j}(t)\\
      && a_{j}(t) &  b_{j+1}(t) & \ddots \\
      &&& \ddots & \ddots
      \end{bmatrix}, ~~ S(t) = \begin{bmatrix}
      \ddots & \ddots\\
      \ddots & 0 & -a_{j-1}(t)\\
      & a_{j-1}(t) & 0 & -a_{j}(t)\\
      && a_{j}(t) &  0 & \ddots \\
      &&& \ddots & \ddots
      \end{bmatrix}
  \end{align*}
  Now, show that \eqref{eq:toda_ab} is equilvalent to
  \begin{align} \label{eq:toda_T}
    T'(t) = S(t)T(t) - T(t)S(t).
  \end{align}
  Hint: Fix $j$ and consider
  \begin{align*}
    e_j^TS(t)T(t)e_k \text{ and } e_j^TT(t)S(t)e_k, \quad k = j-2,j-1,j,j+1,j+2,
  \end{align*}
  where $e_j$ denotes the standard basis vector that is all zero except for a one in the $j$th entry.  Despite the fact that the matrices are bi-infinite, you only need to track the $(j-1)$th, $j$th  and $(j+1)$th entries of vectors such as  $e_j^TS(t)$ and $T(t)e_k$.
  
\mline
  
\item One finite-dimensional approximation of \eqref{eq:toda_T} is to just take a finite section (a square subblock on the diagonal) of both $T$, $P$:
  \begin{align*}
    T_N(t) &= \begin{bmatrix}
      b_1(t) & a_1(t)\\
      a_1(t) & b_{2}(t) & a_{2}(t)\\
      & a_{2}(t) & b_{3}(t) & \ddots\\
      && \ddots & \ddots & a_{N-1}(t) \\
       &&& a_{N-1}(t) & b_N(t)
     \end{bmatrix},\\
    S_N(t) &= \begin{bmatrix}
      0 & a_1(t)\\
      -a_1(t) & 0 & a_{2}(t)\\
      & -a_{2}(t) & 0 & \ddots\\
      && \ddots & \ddots & a_{N-1}(t) \\
       &&& -a_{N-1}(t) & 0.
      \end{bmatrix}
  \end{align*}
  The finite section choice  can be understood by formally setting $q_0 = -\infty$ and $q_{N+1} = + \infty$ and then performing the change of variables \eqref{eq:toda_cov}.
  
  With initial conditions $b_j(0) = 0$, $a_j(0) = 1/2, j = 1,2,\ldots,N$ and $N = 6$, use your favorite time-stepping method to solve
  \begin{align*}
    T_N'(t) = S_N(t) T_N(t) - T_N(t)S_N(t),
  \end{align*}
  to $t = 100$ and plot the solution.  You should notice something striking about the solution. You might want to look at eigenvalues of $T_N(0)$. Comment on this.    Repeat this with $b_j(0) = -2$ and $a_j(0) = 1$, $j = 1,2,\ldots,N$ and $N = 12$.

\mline  

\item The Toda lattice in the finite-section case is a Hamiltonian system with Hamiltonian
  \begin{align*}
    H(p,q) = \frac 1 2 p_N^2 + \sum_{j=1}^{N-1}\left[ \frac 1 2 p_j^2 + e^{-(q_{j+1} - q_j)} \right].
  \end{align*}
  This means that the equations of motion for $q_j(t)$ and $p_j(t)$ can also be written as
  \begin{align}
    \label{eq:pq}
    \begin{split}
    p_j'(t)  &= - \frac{\partial H}{\partial q_j} (p(t),q(t)),\\
    q_j'(t)  & = \frac{\partial H}{\partial p_j} (p(t),q(t)),
  \end{split}\end{align}
 where $ p(t) = (p_j(t))_{j=1}^N$ and $q(t) = (q_j(t))_{j=1}^N$.   And, by the chain rule, $H$ is conserved:
  \begin{align*}
    \frac{d}{dt} H(p(t),q(t)) = 0.
  \end{align*}
  Sympletic numerical integrators for Hamiltonian systems are designed to preserve conserved quantities and geometric properties of systems they approximate.  We can summarize the system \eqref{eq:pq} as
  \begin{align*}
    p'(t) = J(q(t)),  \quad q'(t) = K(p(t)).
  \end{align*}
  One symplectic method is the so-called St\"ormer--Verlet method and it is given by
  \begin{align*}
    P^* &= P^n + \frac{k}{2} J(Q^n),\\
    Q^{n+1} &= Q^n + k K(P^*),\\
    P^{n+1} &= P^* + \frac{k}{2} J(Q^{n+1}).
  \end{align*}
  Convert the initial data $b_j(0) = 0$ and $a_j(0) = 1/2$ for $j = 1,2,\ldots,N$ to $q_j(0), p_j(0)$ for $j = 1,2,\ldots,N$ and solve the system with the St\"ormer-Verlet method.  Perform a convergence study at $t = 1$ for time steps $k = 2^{-j}$, $j = 1,2,3,4,5,6$ (see \url{https://github.com/trogdoncourses/amath-586-2020/blob/master/notebooks/Astability.ipynb}) to determine the order of the method. \\

 \noindent Some hints:
  \begin{itemize}
  \item Since the $a_j$,$b_j$ variables depend only on a difference of the $q_j$.  You can set one value, say $q_1(0)$, to be whatever value you wish.
    \item You also might want to write a function to convert between $a_j,b_j$ and $p_j,q_j$.  Here is a {\tt Julia} implementation:
      \begin{lstlisting}
to_a = (p,q) -> .5*exp.(-(q[2:end]-q[1:end-1])/2)
to_b = (p,q) -> -.5*p
to_p = (a,b) -> -2*b
function to_q(a,b) # chooses q[1] = 0
    q = fill(0.,length(b))
    q[2:end] = -2*log.(2*a)
    cumsum(q)
end \end{lstlisting}
\item Here is a {\tt Julia} implementation of $J$ and $K$:
  \begin{lstlisting}
function J(q) 
    out = fill(0.,length(b))
    temp = exp.(q[1:end-1] - q[2:end])
    out[1:end-1] -= temp
    out[2:end] += temp
    out
end

function K(p)
    p
end
\end{lstlisting}
\end{itemize}

\noindent \underline{Just for your information}:  A second order method will satisfy:
\begin{align*}
  \text{error at time } T \sim C_T k^2.
\end{align*}
And the constant $C_T$ is incredibly important as $T$ increases.  Symplectic methods can be used to keep $C_T$ from growing too rapidly and they are very important in, say, planetary dynamics over long time scales.
      

\mline
      
\item (Extra credit)  Form a finite-dimensional approximation of \eqref{eq:toda} using the boundary condition $q_0(t) = q_{N+1}(t)$  and performing the change of variables \eqref{eq:toda_cov}.  Update the matrices $T_N$ and $S_N$ for the periodic case.
  
  \end{enumerate}

\end{document}

%%% Local Variables:
%%% mode: latex
%%% TeX-master: t
%%% End:
